\section{Beschreibung des User Interface Konzepts}

[Detaillierte Beschreibung des Gesamtkonzepts: Wie wird das Interaktionsverhalten abgebildet? Wieso wurde diese Lösung gewählt? Wie wurden die Personas und Szenarios in der Konzeptausarbeitung berücksichtigt? Welche besonderen Prinzipien wurden bei der Realisierung beachtet?
Darstellung anhand der Abbildungen aller erstellten Wireframes oder Mockups und ggf. der Interaktionsabläufe als Flussdiagramm inkl. Beschreibung bzw. Erläuterung der dahinter stehenden Überlegungen in den folgenden Unterkapiteln. Der Zusammenhang der einzelnen Bildschirmmasken und Interaktionsfluss durch das Programm soll klar hervorgehen. ]

\subsection{Desktop Web Interface}
\subsection{Mobiles User Interface}
\subsection{Detailfragen}

\subsubsection{Frage 1}

\emph{Wie kann man Benutzer bei der Eingabe von Datensätzen unterstützen? Wie lassen sich Fehleingaben vermeiden? Wie werden diese Konzepte in Ihrem Prototypen realisiert?}


[Max. 250 Wörter Wireframes zur Illustration]



\subsubsection{Frage 2}

\emph{Welche Informationen sind für Benutzer besonders wichtig und wie lässt sich deren Bedeutung im System repräsentieren? Welche Such-/Filter-/Sortier-Funktionen sind nützlich?}



[Max. 250 Wörter Wireframes zur Illustration]



\subsubsection{Frage 3}

\emph{Welche Möglichkeiten gibt es zur graphischen Aufbereitung der Daten (Grafiken, Kalender-Ansicht, etc.)? Wie werden diese Möglichkeiten in Ihrem Prototypen realisiert?}



[Max. 250 Wörter Wireframes zur Illustration]



\subsubsection{Frage 4}

\emph{Wie lässt sich eine sinnvolle Aufgabenteilung zwischen Desktop-UI und mobiler UI erreichen? Welche Aufgaben haben bei der Verwendung zu Hause am Schreibtisch die höchste Priorität und welche Aufgaben haben bei der Verwendung am Smartphone die höchste Priorität? Welche Formen der graphischen Aufbereitung sind für den jeweiligen Anwendungskontext angemessen?}



[Max. 250 Wörter Wireframes zur Illustration]
