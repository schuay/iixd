\section{Personas und Szenarios}

\subsection{Personas}

\subsubsection{Horst}

\begin{wrapfigure}{r}{4.5cm}
\centering
\includegraphics[width=4cm]{img/horst}
\caption{Horst und Elfriede}
\label{fig:horst_und_elfriede}
\end{wrapfigure}

\textbf{Name}: Horst Jodocus Kwak \\
\textbf{Beruf}: Bankkaufmann \\
\textbf{Alter}: 31 Jahre \\
\textbf{Geschlecht}: M\"annlich \\
\textbf{Beziehungsstatus}: frisch verheiratet \\
\textbf{Kinder}: aktuell 0, in zwei Monaten 1 \\

Horst ist ein typischer junger Familienvater. Seine Frau kennt er seit der Schulzeit,
und nach langj\"ahrige Beziehung haben sie vor zwei Monaten geheiratet.

Als Bankkaufmann
hat Horst ein annehmliches und fixes Einkommen. Trotzdem sind seine Finanzen au\ss erordentlich eingeschr\"ankt;
einerseits hat sich und seiner Familie Horst vor kurzem ein kleines Reihenhaus gekauft (wodurch seine
monatlichen Kosten betr\"achtlich gestiegen sind), und andererseits muss er schon in Anbedacht auf das neue
Familienmitglied planen und sparen.

Horst hat durch seinen Beruf regelm\"a\ss ig Kontakt mit Computern, f\"uhlt sich aber schnell verwirrt, hilflos und
frustriert sobald etwas nicht so funktioniert wie er will, oder er sich in neuen Anwenderprogrammen zurechtfinden will.
Sein Smartphone bereitet ihm dagegen mehr Freude; er spielt vor allem gerne damit w\"ahrend
er von anderen Leuten gesehen wird.

Seine Ziele sind simpel: Horst will mit seinen Mitteln nicht nur die laufenden Kosten des Hauses,
die kommenden Kosten des Kindes und alle restlichen regelm\"a\ss ig anfallende Ausgaben bew\"altigen,
sondern auch sich ab und zu eine Kleinigkeit (wie zum Beispiel das neueste Smartphone Modell) leisten k\"onnen.

\newpage
\subsubsection{Frauke}

\begin{wrapfigure}{r}{4.5cm}
\centering
\includegraphics[width=4cm]{img/frauke}
\caption{Frauke}
\label{fig:frauke}
\end{wrapfigure}

\textbf{Name}: Frauke van Fraukenstein \\
\textbf{Beruf}: Studentin \\
\textbf{Alter}: 22 Jahre \\
\textbf{Geschlecht}: Weiblich \\
\textbf{Beziehungsstatus}: Single \\
\textbf{Kinder}: keine \\

Frauke ist Studentin an der Universitaet Wien, Hauptfach Psychologie im 2.Abschnitt.
Da sie gerne in der Natur unterwegs und auch noch leidenschaftliche Bergsteigerin ist,
benoetigt sie ein Auto, um schnell und einfach zu den nahegelegenen Bergen gelangen
zu koennen. Um die benoetigte Summe zum Kauf eines Autos aufbringen zu koennen, muss
sie an allen Ecken und Enden sparen. Da ihr der Ueberblick ueber ihre im Studentenleben
zu taetigenden Ausgaben leicht verloren geht, waere eine unterstuetzende Software sehr
hilfreich. Wie die meisten ihre AltersgenossInnen besitzt sie ein Smartphone, welches
sie staendig mit sich traegt. Ideal waere deshalb eine Anwendung die es ihr ermoeglicht,
schnell und einfach, gerade getaetigte Ausgaben auf ihrem SmartPhone zu erfassen und
gegebenfalls zu Hause, bequem per Web-Applikation auszuwerten.

\subsection{Szenarios}

\subsubsection{Horst}

Horst macht sich nach einem arbeitsreichen Tag auf den Heimweg. Unterwegs muss er zum
Postamt um ein Paket abzuholen. Zuvor geht er aber in ein Elektronikgesch\"aft.
Elfriede m\"ochte ein neues Smartphone.\\

Nach einigem Herumgesuche findet Horst schlie\ss lich das von ihr gew\"unschte
Modell, welches einen stattlichen Preis aufweist. Auf seinem eigenen
Smartphone gibt er den Betrag ein und sieht sofort, dass dieser, zusammen
mit den Kreditr\"uckzahlungen und Betriebskosten ihres Reihenhauses und
einigen anderen Ausgaben, die diesen Monat noch anfallen werden die
Ausgabengrenze, die er und Elfriede sich gesetzt haben \"ubersteigen wird.
Seine Frau wird wohl noch ein Monat auf ihr neues Smartphone warten
m\"ussen.\\

Beim verlassen des Gesch\"afts erinnert Horst sich daran, dass er neue
Tintenpatronen für seinen Drucker kaufen wollte. Er st\"urmt wieder hinein
und kauft 1x Schwarz und 1x Cyan. Den Betrag der Rechnung gibt er sofort
auf seinem Smartphone ein und wirft sie danach weg.\\

Zuhause angekommen packt Horst zusammen mit Elfriede das Paket aus (er hat
nat\"urlich nicht vergessen es abzuholen, Horst vergisst nicht). Nachdem die
riesige Ladung an Strampelanz\"ugen, Bilderb\"uchern und Spielzeug verstaut
ist und Horst die Comics die er bestellt hat zum Rest seiner Sammlung
gestellt hat, setzt er die neuen Druckerpatronen ein und sich vor den PC.
Er tr\"agt die, dem Paket beiliegende, Rechnung ein, wobei er aber das Geld
für seine Comics in die eigens daf\"ur vorgesehene Kategorie eintr\"agt.
Au\ss erdem tr\"agt er die Anschaffungen für das Kind ebenfalls in eine eigene
Kategorie ein.\\

Horst entschlie\ss t sich auch noch einige andere Rechnungen, die noch auf
seinem Schreibtisch herumliegen einzutragen. Dann sieht er sich an wof\"ur er
und Elfriede diesen Monat am meisten ausgegeben haben. Die neue Wohnung und
das erwartete Baby f\"uhren die Liste an. Allerdings sieht es so aus, als ob
sie es schaffen werden, noch ein wenig Geld beiseite zu legen. Vielleicht
wird es ja n\"achstes Monat was mit Elfriedes neuem Smartphone.\\

\subsubsection{Frauke}

Frauke hat gerade an der Kasse nach einem Lebensmitteleinkauf gezahlt und
sollte nun moeglichst schnell und einfach die Einkaufssumme in die
Ausgabenerfassungsapplikation auf ihrem Smartphone eintragen koennen. Die 
noetige Eingabe begrenzt sich auf Summe und Kategorie, optional kann sie noch 
zusaetzliche Informationen hinzufuegen, wie z.B. Kurze Beschreibung, Geschaeft, 
usw. Ebenfalls ist es auch moeglich das sie die Ausgaben erst zu Hause am PC 
eintraegt. Hier kann sie auch ihre wiederkehrenden Ausgaben verwalten und eine 
detailierte Auswertung durchfuehren.\\

[Beschreibung der entwickelten Szenarios: Textuelle Beschreibung der Szenarios in Hinblick auf den Einsatz Ihres Konzepts, ggf. ergänzt durch Skizzen o.ä.]
