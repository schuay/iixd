\section{Personas und Szenarios}

\subsection{Personas}

\subsubsection{Horst}

\begin{wrapfigure}{r}{4.5cm}
\centering
\includegraphics[width=4cm]{img/horst}
\caption{Horst und Elfriede}
\label{fig:horst_und_elfriede}
\end{wrapfigure}

\textbf{Name}: Horst Jodocus Kwak \\
\textbf{Beruf}: Bankkaufmann \\
\textbf{Alter}: 31 Jahre \\
\textbf{Geschlecht}: M\"annlich \\
\textbf{Beziehungsstatus}: frisch verheiratet \\
\textbf{Kinder}: aktuell 0, in zwei Monaten 1 \\

Horst ist ein typischer junger Familienvater. Seine Frau kennt er seit der Schulzeit,
und nach langj\"ahrige Beziehung haben sie vor zwei Monaten geheiratet.

Als Bankkaufmann
hat Horst ein annehmliches und fixes Einkommen. Trotzdem sind seine Finanzen au\ss erordentlich eingeschr\"ankt;
einerseits hat sich und seiner Familie Horst vor kurzem ein kleines Reihenhaus gekauft (wodurch seine
monatlichen Kosten betr\"achtlich gestiegen sind), und andererseits muss er schon in Anbedacht auf das neue
Familienmitglied planen und sparen.

Horst hat durch seinen Beruf regelm\"a\ss ig Kontakt mit Computern, f\"uhlt sich aber schnell verwirrt, hilflos und
frustriert sobald etwas nicht so funktioniert wie er will, oder er sich in neuen Anwenderprogrammen zurechtfinden will.
Sein Smartphone bereitet ihm dagegen mehr Freude; er spielt vor allem gerne damit w\"ahrend
er von anderen Leuten gesehen wird.

Seine Ziele sind simpel: Horst will mit seinen Mitteln nicht nur die laufenden Kosten des Hauses,
die kommenden Kosten des Kindes und alle restlichen regelm\"a\ss ig anfallende Ausgaben bew\"altigen,
sondern auch sich ab und zu eine Kleinigkeit (wie zum Beispiel das neueste Smartphone Modell) leisten k\"onnen.

\subsubsection{Frauke}

\subsection{Szenarios}

[Beschreibung der entwickelten Szenarios: Textuelle Beschreibung der Szenarios in Hinblick auf den Einsatz Ihres Konzepts, ggf. ergänzt durch Skizzen o.ä.]
