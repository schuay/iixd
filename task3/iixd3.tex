\documentclass[a4paper,10pt]{article}
\usepackage[utf8]{inputenc}
\usepackage{graphicx}
\usepackage{float}
\usepackage{comment}
\usepackage[ngerman]{babel}
\usepackage[pdfborder={0 0 0}]{hyperref}
\usepackage{lmodern,textcomp}
\usepackage{subfigure}

\title{
    183.289 VU Interface \& Interaction Design \\
    WS 2012 / 2013 \\
    Beispiel 3}
\author{
    Gruppe 5 \\ \\
    Jakob Gruber, 0203440, 033 534 \\
    Rene Pointner, 0526792, 033 532\\
    Mino Sharkhawy, 1025887, 033 534 \\
    Stefan Sietzen, 0372194, 033 532}

\begin{document}

\maketitle

\clearpage
\tableofcontents

\clearpage
\section{Umsetzung des Prototypen}

Der interaktive Prototyp (in diesem Dokument IP genannt) \emph{Finance Buddy} wurde mit Hilfe von Twitter
Bootstrap\footnote{\url{http://twitter.github.com/bootstrap/index.html}}
und jQuery-UI\footnote{\url{http://jqueryui.com/} und
\url{http://addyosmani.github.com/jquery-ui-bootstrap/}} in HTML5 realisiert.

Als Vorlage wurde der angepasste Low-Fidelity-Prototype der zweiten Aufgabe von Interface \&
Interaction Design (in diesem Dokument LFP genannt) verwendet. Es wurden dabei einzelne Screens nacheinander als
entsprechende HTML Seiten erstellt. Kleinere, von einander getrennte Screens mit
wenig gemeinsamen Elementen (wie zum Beispiel Login, Registration, und Password Recovery)
sind als separate HTML Seiten gehalten worden, w\"ahrend der eigentliche Hauptteil
der Applikation in einer einzigen Seite realisiert wurde.

Der Gedanke dahinter war vor allem Arbeitsersparnis. Durch viele gemeinsame Elemente (unter
anderem die gesamte Leiste auf der linken Seite des Bildschirms) h\"atte sich der
Arbeitsaufwand vervielfacht wenn die jeweiligen Untersektionen (\emph{Expenses},
\emph{Recurring Expenses}, \emph{Graphs}, etc\ldots) separate HTML Seiten w\"aren.

Die L\"osung st\"utzt sich somit auf JavaScript um UI Elemente dynamisch ein- und
auszublenden oder anderwertig zu ver\"andern. Zum Beispiel wird bei einem Click auf
den \emph{Recurring Expenses} Eintrag die passende rechte Seite des Screens eingeblendet,
die linke Seite wird mit den richtigen Elementen bef\"ullt, und der Navigationseintrag
wird als aktiv markiert.

\subsection{Abweichungen vom urspr\"unglichen Konzept}

\begin{comment}
Beschreiben Sie kurz die Umsetzung Ihres interaktiven Prototypen. Insbesondere sollen darin folgende Fragen beantwortet werden (jeweils soweit sie auf Ihren Entwicklungsprozess anwendbar sind):

Welche \"Anderungen waren bei der Umsetzung des interaktiven Prototypen im Vergleich zu Ihrem urspr\"unglichen Konzept notwendig und weshalb (technische Einschr\"ankungen, Komplexit\"at, ...)?

Welche Aspekte waren in Ihrem urspr\"unglichen Konzept unterspezifiziert und mussten bei der Umsetzung konkretisiert werden?

Auf welche Schw\"achen und Probleme in Ihrem Konzept sind Sie bei der Umsetzung des interaktiven Prototypen aufmerksam geworden?


Illustrieren Sie Abweichungen anhand von Wireframes aus Ihrem urspr\"unglichen Konzept und stellen Sie sie Screenshots aus dem interaktiven Prototypen gegen\"uber. Heben Sie ggf. spezifische Details graphisch oder in Form von Anmerkungen hervor.
\end{comment}

\clearpage
\subsubsection{Sign In}

\begin{figure}
\centering
\includegraphics[width=\textwidth]{sign-in}
\caption{Sign In} \label{fig:sign-in}
\end{figure}

Der Sign In Screen ist von den Funktionen her identisch zum LFP. Es wurden mehrere
kleinere \"Anderungen an dem Look \& Feel vorgenommen: wichtige Elemente wie der
Sign In Button wurden durch Typographie und Farbgestaltung hervorgehoben, und das Layout
ist nun im IP sinnvoller gestaltet.

Bei falschen Eingaben (zum Beispiel einem leeren Username Feld) wird dem Benutzer
deutlicher Feedback in Form eines roten Alerts angezeigt.

\clearpage
\subsubsection{Password Recovery}

\begin{figure}
\centering
\includegraphics[width=\textwidth]{password-recovery}
\caption{Password Recovery} \label{fig:password-recovery}
\end{figure}

Das Layout ist im IP um einiges simpler gestaltet als im LFP. Wichtige Elemente
sind hervorgehoben und Zusatzinfos zu n\"achsten Schritten nach Ausf\"ullen
des Formulars werden erst angezeigt wenn sie relevant sind.

\clearpage
\subsubsection{Registration}

\begin{figure}
\centering
\includegraphics[width=\textwidth]{registration}
\caption{Registration} \label{fig:registration}
\end{figure}

Bei dem Registrierungsinterface wurden die Platzraubenden Labels links neben den Textfeldern entfernt, 
und komplett durch Placeholdertexte in den Textfeldern ersetzt. Das f\"uhrt zu einer harmonischeren \"Asthetik und spart Platz.

\clearpage
\subsubsection{Advanced Search}

\begin{figure}
\centering
\includegraphics[width=\textwidth]{advanced-search}
\caption{Advanced Search} \label{fig:advanced-search}
\end{figure}

Der Advanced Search Dialog ist vom Aufbau her nur gering umgestaltet worden. Es ist nur der Search-Button vom Rand in die
Mitte verschoben worden. Die Funktionen sind gleich dem LFD.

\clearpage
\subsubsection{Expenses \& Recurring Expenses}

\begin{figure}
\centering
\includegraphics[width=\textwidth]{expenses}
\caption{Expenses \& Recurring Expenses} \label{fig:expenses}
\end{figure}

\begin{figure}
\centering
\includegraphics[width=\textwidth]{expenses-new-ip}
\caption{New Expense} \label{fig:expenses-new}
\end{figure}

Die wichtigste \"Anderung ist das Herausnehmen der Eingabezeile in ein eigenes Popup,
das durch den \glqq Add New Expense\grqq\space Button
ge\"offnet wird. Das f\"uhrt zu einer verbesserten \"Ubersichtlichkeit und zu einer verst\"andlicheren Bedienung.
In dem Popup hat man nach Eingabe der Daten wahlweise die M\"oglichkeit, weitere Datens\"atze einzugeben oder das Popup wieder
zu schlie\ss en. Au\ss erdem wurde die Bearbeitung der wiederkehrenden Ausgaben aus dem
\glqq Expenses\grqq\space Interface entfernt und daf\"ur
ein eigener Men\"upunkt \glqq Recurring Expenses\grqq\space angelegt. Dort k\"onnen bequem alle wiederkehrenden Ausgaben betrachtet und
bearbeitet, sowie neue Ausgaben eingef\"ugt werden. Der Unterschied zum
Eingabefenster f\"ur \glqq normale\grqq\space Ausgaben ist,
dass man Anstatt eines \glqq Dates\grqq\space ein \glqq Start Date\grqq\space
festlegt und zus\"atzlich ein Intervall.
Weiters werden nur noch die Ausgaben eines Monats aufgelistet. Man kann diesen in der
unteren Leiste ausw\"ahlen. Dadurch f\"allt
die Zusammenfassungszeile nach jedem Monat in der Liste weg, stattdessen wird links unten in einem Kasten die Zusammenfassung
f\"ur den ausgew\"ahlten Monat inklusive Farbbalken (rot bei \"Uberschreitung, gr\"un bei noch verf\"ugbarem Betrag) dargestellt.
Das Interface ist dadurch wesentlich \"ubersichtlicher geworden.

\clearpage
\subsubsection{Settings}

\begin{figure}
\centering
\includegraphics[width=\textwidth]{settings}
\caption{Settings} \label{fig:settings}
\end{figure}

Der Settings Screen wurde vom Aufbau her etwas umgestaltet, ist jedoch von den Funktionen her derselbe wie im LFD. Anders ist 
der vorgegebene Wert im Budgetlimit Eingabefeld, welcher vom Text \glqq
Limit\grqq\space durch einen vorgegebenen Wert von \glqq 1500.00\grqq\space ersetzt
wurde. Falsche Eingaben (wie Buchstaben oder Betrag kleiner Null) werden erkannt und es wird eine entsprechende Fehlermeldung ausgegeben.


\clearpage
\subsection{Einschr\"ankungen des interaktiven Prototypen}

\begin{comment}
Welche Aspekte konnten in Ihrem interaktiven Prototypen nur unvollst\"andig umgesetzt
werden (z.B. weil der Umsetzungsaufwand zu hoch w\"are)? Beschreiben Sie eventuelle
Kompromissl\"osungen ausreichend genau, dass Ihre geplante und beabsichtigte
Funktionsweise f\"ur den Leser klar verst\"andlich und nachvollziehbar ist.
\end{comment}

\begin{itemize}
    \item Registrierung, Login und Password Recovery testen die Eingaben zwar auf
            Korrektheit, haben aber keine echte Nutzerdatenbank. Der Login akzeptiert
            jede Benutzername/Passwort-Kombination.
    \item Das Sortieren nach Spalten unter \glqq Expenses\grqq, \glqq
            Recurring Expenses\grqq\space und dem Suchergebnis ist mit dem Prototypen
            nicht m\"oglich.
    \item Der \glqq Add New Recurring Expense\grqq\space Dialog f\"uhrt
            Fehler\"uberpr\"ufung durch, es ist aber nicht m\"oglich einen neuen
            Eintrag hinzuzuf\"ugen.
    \item Die \glqq Edit\grqq\space und \glqq Remove\grqq\space Buttons unter \glqq
            Recurring Expenses\grqq\space funktionieren wegen des hohen Aufwands
            nicht. Der \glqq Edit\grqq\space Dialog sieht aus wie der zum
            Hinzuf\"ugen und wird unter der entsprechenden Zeile angezeigt.
    \item Die Suche und erweiterte Suche liefern keine echten Ergebnisse sondern nur
            eine statische Seite.
    \item Die Buttons \glqq August 2012\grqq, \glqq July 2012\grqq\space und \glqq
            June 2012\grqq\space unter \glqq Expenses\grqq\space haben keine
            Funktion.
    \item Die Einstellungen unter \glqq Settings\grqq\space wirken sich nicht auf
            andere Anzeigen aus und gehen nachdem die Seite neu geladen wurde
            verloren.
    \item Unabh\"angig davon welcher Zeitraum oder welche Kategorie unter \glqq
            Graphs\grqq\space ausgew\"ahlt wurde, gibt es nur zwei Graphen als
            Bilder.
    \item Unter \glqq Expenses\grqq\space kann man mit dem \glqq Add New
            Expense\grqq\space Dialog genau einen neuen Eintrag hinzuf\"ugen indem
            man nach Eingabe der Daten auf \glqq Submit\grqq\space klickt. \glqq
            Submit and Continue\grqq\space f\"ugt keinen Datensatz hinzu. Der neue
            Eintrag wird \"uber GET-Parameter gesichert und erscheint im Suchergebnis
            und unter \glqq Expenses\grqq\space jeweils an erster Stelle. F\"ur
            diesen und nur diesen Eintrag funktionieren die \glqq Edit\grqq\space und
            \glqq Remove\grqq\space Buttons.
    \item Wenn ein funktionierender \glqq Submit\grqq\space Button gedr\"uckt wird
            und die Seite neu geladen wird landet man wieder unter \glqq
            Expenses\grqq\space auf der Seite f\"ur Oktober 2012. Tats\"achlich
            sollte man aber auf der selben Seite bleiben. Das ist aber relativ
            schwierig und komplex umzusetzen.
    \item Beim Eingeben einer Kategorie (\glqq Category\grqq-Feld) sollten nicht nur
            Vorschl\"age angezeigt werden, sondern auch bereits angelegte Kategorien
            \"uber ein Drop-Down-Men\"u ausgew\"ahlt werden k\"onnen. Das war aber
            mit den gegebenen Mitteln zu aufw\"andig.
    \item Der Beispieleintrag, den man wie oben erkl\"art anlegen kann, ver\"andert
            weder die Grafiken unter \glqq Graphs\grqq, noch den Balken und die Daten
            in der Zusammenfassung links unten.
\end{itemize}

\end{document}
