\documentclass[a4paper,10pt]{article}
\usepackage[utf8]{inputenc}
\usepackage{graphicx}
\usepackage{float}
\usepackage{comment}
\usepackage[ngerman]{babel}
\usepackage[pdfborder={0 0 0}]{hyperref}
\usepackage{lmodern,textcomp}

\title{
    183.289 VU Interface \& Interaction Design \\
    WS 2012 / 2013 \\
    Beispiel 3}
\author{
    Gruppe 5 \\ \\
    Jakob Gruber, 0203440, 033 534 \\
    Rene Pointner, 0526792, 033 532\\
    Mino Sharkhawy, 1025887, 033 534 \\
    Stefan Sietzen, 0372194, 033 532}

\begin{document}

\maketitle

\clearpage
\tableofcontents

\clearpage
\section{Umsetzung des Prototypen}
\subsection{Abweichungen vom urspr\"unglichen Konzept}

\begin{comment}
Beschreiben Sie kurz die Umsetzung Ihres interaktiven Prototypen. Insbesondere sollen darin folgende Fragen beantwortet werden (jeweils soweit sie auf Ihren Entwicklungsprozess anwendbar sind):

Welche Änderungen waren bei der Umsetzung des interaktiven Prototypen im Vergleich zu Ihrem ursprünglichen Konzept notwendig und weshalb (technische Einschränkungen, Komplexität, ...)?

Welche Aspekte waren in Ihrem ursprünglichen Konzept unterspezifiziert und mussten bei der Umsetzung konkretisiert werden?

Auf welche Schwächen und Probleme in Ihrem Konzept sind Sie bei der Umsetzung des interaktiven Prototypen aufmerksam geworden?


Illustrieren Sie Abweichungen anhand von Wireframes aus Ihrem ursprünglichen Konzept und stellen Sie sie Screenshots aus dem interaktiven Prototypen gegenüber. Heben Sie ggf. spezifische Details graphisch oder in Form von Anmerkungen hervor.
\end{comment}

\clearpage
\subsection{Einschränkungen des interaktiven Prototypen}

\begin{comment}
Welche Aspekte konnten in Ihrem interaktiven Prototypen nur unvollständig umgesetzt werden (z.B. weil der Umsetzungsaufwand zu hoch wäre)? Beschreiben Sie eventuelle Kompromisslösungen ausreichend genau, dass Ihre geplante und beabsichtigte Funktionsweise für den Leser klar verständlich und nachvollziehbar ist.
\end{comment}

\end{document}
