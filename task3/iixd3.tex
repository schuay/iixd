\documentclass[a4paper,10pt]{article}
\usepackage[utf8]{inputenc}
\usepackage{graphicx}
\usepackage{float}
\usepackage{comment}
\usepackage[ngerman]{babel}
\usepackage[pdfborder={0 0 0}]{hyperref}
\usepackage{lmodern,textcomp}

\title{
    183.289 VU Interface \& Interaction Design \\
    WS 2012 / 2013 \\
    Beispiel 3}
\author{
    Gruppe 5 \\ \\
    Jakob Gruber, 0203440, 033 534 \\
    Rene Pointner, 0526792, 033 532\\
    Mino Sharkhawy, 1025887, 033 534 \\
    Stefan Sietzen, 0372194, 033 532}

\begin{document}

\maketitle

\clearpage
\tableofcontents

\clearpage
\section{Umsetzung des Prototypen}

Der interaktive Prototyp \emph{Finance Buddy} wurde mit Hilfe von Twitter
Bootstrap\footnote{\url{http://twitter.github.com/bootstrap/index.html}}
und jQuery-UI\footnote{\url{http://jqueryui.com/} und
\url{http://addyosmani.github.com/jquery-ui-bootstrap/}} in HTML5 realisiert.

Als Vorlage wurde der angepasste Low-Fidelity-Prototype der zweiten Aufgabe von Interface \&
Interaction Design verwendet. Es wurden dabei einzelne Screens nacheinander als
entsprechende HTML Seiten erstellt. Kleinere, von einander getrennte Screens mit
wenig gemeinsamen Elementen (wie zum Beispiel Login, Registration, und Password Recovery)
sind als separate HTML Seiten gehalten worden, w\"ahrend der eigentliche Hauptteil
der Applikation in einer einzigen Seite realisiert wurde.

Der Gedanke dahinert war vor allem Arbeitsersparnis. Durch viele gemeinsame Elemente (unter
anderem die gesamte Leiste auf der linken Seite des Bildschirms) h\"atte sich der
Arbeitsaufwand vervielfacht wenn die jeweiligen Untersektionen (\emph{Expenses},
\emph{Recurring Expenses}, \emph{Graphs}, etc\ldots) separate HTML Seiten waeren.

Die L\"osung st\"utzt sich somit auf JavaScript um UI Elemente dynamisch ein- und
auszublenden oder anderwertig zu ver\"andern. Zum Beispiel wird bei einem Click auf
den \emph{Recurring Expenses} Entrag die passende rechte Seite des Screens eingeblendet,
die linke Seite wird mit den richtigen Elementen bef\"ullt, und der Navigationseintrag
wird als aktiv markiert.

\subsection{Abweichungen vom urspr\"unglichen Konzept}

\begin{comment}
Beschreiben Sie kurz die Umsetzung Ihres interaktiven Prototypen. Insbesondere sollen darin folgende Fragen beantwortet werden (jeweils soweit sie auf Ihren Entwicklungsprozess anwendbar sind):

Welche Änderungen waren bei der Umsetzung des interaktiven Prototypen im Vergleich zu Ihrem ursprünglichen Konzept notwendig und weshalb (technische Einschränkungen, Komplexität, ...)?

Welche Aspekte waren in Ihrem ursprünglichen Konzept unterspezifiziert und mussten bei der Umsetzung konkretisiert werden?

Auf welche Schwächen und Probleme in Ihrem Konzept sind Sie bei der Umsetzung des interaktiven Prototypen aufmerksam geworden?


Illustrieren Sie Abweichungen anhand von Wireframes aus Ihrem ursprünglichen Konzept und stellen Sie sie Screenshots aus dem interaktiven Prototypen gegenüber. Heben Sie ggf. spezifische Details graphisch oder in Form von Anmerkungen hervor.
\end{comment}

\clearpage
\subsection{Einschränkungen des interaktiven Prototypen}

\begin{comment}
Welche Aspekte konnten in Ihrem interaktiven Prototypen nur unvollständig umgesetzt werden (z.B. weil der Umsetzungsaufwand zu hoch wäre)? Beschreiben Sie eventuelle Kompromisslösungen ausreichend genau, dass Ihre geplante und beabsichtigte Funktionsweise für den Leser klar verständlich und nachvollziehbar ist.
\end{comment}

\end{document}
